\documentclass[10pt]{examdesign}
\usepackage[spanish]{babel} % Paquete de idiomas con opción en español
\usepackage[utf8]{inputenc} % Paquete de codificación de caraacteres con opción en utf8
\usepackage[T1]{fontenc}    % Paquete de codificación de fuentes con opción en T1
\usepackage{amsmath}
\Fullpages
\NoKey
\examname{Trabajo Práctico}
\class{Nombre de mi clase}
\begin{document}

%-------------------------------             SHORT ANSWER        ------------------------%
\begin{shortanswer}[title={Short Answer (10 pts each)},
	rearrange=yes,resetcounter=no]
	This is an example of the \textsf{shortanswer} type of question, where the
	questions are rearranged between tests.
	
	\begin{question}
		Here is an equation:
		\begin{align}
		f(x) &= h(x)\\
		g(x) &= j(x)
		\end{align}
		`Twas brillig, and the slithy toves  did gyre and gimble in the wabe;
		All mimsy were the borogoves, and the mome raths outgrabe.
%		\InsertChunk{Sample program no. 1}
		\begin{answer}
			Ask Lewis Carroll.
		\end{answer}
	\end{question}
	
	\begin{question}
		`And has thou slain the Jabberwock? Come to my arms, my beamish boy!
		O frabjous day! Calooh! Callay!' He chortled in his joy.
		\begin{answer}
			Ask Lewis Carroll.
		\end{answer}
	\end{question}
	
	\begin{question}
		`Twas brillig, and the slithy toves did gyre and gimble in the wabe;
		all mimsy were the borogoves, and the mome raths outgrabe.
		\begin{answer}
			Ask Lewis Carroll.
		\end{answer}
	\end{question}
	
	\begin{question}
		And as in uffish thought he stood, the Jabberwock, with eyes of flame,
		came whiffling through the tulgey wood, and burbled as it came!
		\begin{answer}
			Ask Lewis Carroll.
		\end{answer}
	\end{question}
	
	\begin{question}
		One, two! One, two! And through and through the vorpal blade went snicker-snack!
		He left it dead, and with its head he went galumphing back.
		\begin{answer}
			Ask Lewis Carroll.
		\end{answer}
	\end{question}
	
	\begin{question}
		`And has thou slain the Jabberwock? Come to my arms, my beamish boy!
		O frabjous day! Calooh! Callay!' He chortled in his joy.
		\begin{answer}
			Ask Lewis Carroll.
		\end{answer}
	\end{question}
	
	\begin{question}
		`Twas brillig, and the slithy toves did gyre and gimble in the wabe;
		all mimsy were the borogoves, and the mome raths outgrabe.
		\begin{answer}
			Ask Lewis Carroll.
		\end{answer}
	\end{question}
	
	\begin{question}
		And as in uffish thought he stood, the Jabberwock, with eyes of flame,
		came whiffling through the tulgey wood, and burbled as it came!
		\begin{answer}
			Ask Lewis Carroll.
		\end{answer}
	\end{question}
	
	
	
	
	
	
	
\end{shortanswer}
\end{document}